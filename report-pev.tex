%% LaTeX2e template for FEUP's PEV & other internshis
%% 
%% FEUP, JCL, Thu Dec 14 15:23:03 2023
%%
%% PLEASE send improvements to jlopes at fe.up.pt (up till 2028 :-)

%% v2025:
%% - added support for SI Unit rules and style conventions
%% - added acronyms using the corresponding package

\documentclass[11pt,a4paper]{report}

%%------------------------------- preamble ------------------------------------

%% comment next for EN
\usepackage[portuges]{babel}      % language PT 
\usepackage[utf8]{inputenc}       % accents
\usepackage[T1]{fontenc}          % PS fonts
\usepackage{newtxtext,newtxmath}  % do not use CM fonts
\usepackage{amsmath}              % multi-line and other mathematical statements
\usepackage{setspace}             % setting the spacing between lines
\usepackage{graphicx}             % go far beyond what the graphics package
\usepackage[normalem]{ulem}       % various types of underlining
\usepackage{caption}              % rotating captions, sideways captions, etc.
\usepackage{float}                % tables and figures in the multi-column environment 
\usepackage{subcaption}           % for subfigures and the like
\usepackage{longtable}            % tables that continue to the next page
\usepackage{multirow}             % tabular cells spanning multiple rows
\usepackage{booktabs}             % to create professional-quality tables
\usepackage[table]{xcolor}        % driver-independent color extensions
\usepackage{lipsum}               % loren dummy text
\setlength{\marginparwidth}{2cm}  % todonotes' requirements
\usepackage{todonotes}            % todo's
\usepackage{csquotes}             % context sensitive quotation facilities
\usepackage[backend=biber,authordate]{biblatex-chicago}  % Chicago Manual of Style
\usepackage{pgfgantt}             % Gantt charts
\usepackage[printonlyused]{acronym} % for acronyms
\usepackage{eurosym}              % for the Euro currency symbol
\usepackage{siunitx}              % for SI Unit rules and style conventions
% \sisetup{
%   group-separator = {\,},         % a thin space
%   group-minimum-digits = 4        % enable grouping for numbers ≥ 1000
% }
\DeclareSIUnit{\eurocurrency}{\text{\euro}}  % for the Euro symbol in siunitex

%% document dimensions
\usepackage[a4paper,left=25mm,right=25mm,top=25mm,bottom=25mm,headheight=6mm,footskip=12mm]{geometry}
\setlength{\parindent}{0em}
\setlength{\parskip}{1ex}

%% headers & footers
\usepackage{lastpage}
\usepackage{fancyhdr}
\fancyhf{}                            % clear off all default fancyhdr headers and footers
\rhead{\small{\emph{\projtitle, \projauthor}}}
\rfoot{\small{\thepage\ / \pageref{LastPage}}}
\pagestyle{fancy}                     % apply the fancy header style
\renewcommand{\headrulewidth}{0.4pt}
\renewcommand{\footrulewidth}{0.4pt}

%% colors
\usepackage{color}
\definecolor{engineering}{rgb}{0.549, 0.176, 0.098}
\definecolor{cloudwhite}{cmyk}{0,0,0,0.025}

%% source-code listings
\usepackage{listings}
\lstset{ %
 language=C,                        % choose the language of the code
 basicstyle=\footnotesize\ttfamily,
 keywordstyle=\bfseries,
 numbers=left,                      % where to put the line-numbers
 numberstyle=\scriptsize\texttt,    % the size of the fonts that are used for the line-numbers
 stepnumber=1,                      % the step between two line-numbers. If it's 1 each line will be numbered
 numbersep=8pt,                     % how far the line-numbers are from the code
 frame=tb,
 float=htb,
 aboveskip=8mm,
 belowskip=4mm,
 backgroundcolor=\color{cloudwhite},
 showspaces=false,                  % show spaces adding particular underscores
 showstringspaces=false,            % underline spaces within strings
 showtabs=false,                    % show tabs within strings adding particular underscores
 tabsize=2,                         % sets default tabsize to 2 spaces
 captionpos=t,                      % sets the caption-position to top
 belowcaptionskip=12pt,             % space between caption and listing
 breaklines=true,                   % sets automatic line breaking
 breakatwhitespace=false,           % sets if automatic breaks should only happen at whitespace
 escapeinside={\%*}{*)},            % if you want to add a comment within your code
 morekeywords={*,var,template,new}  % if you want to add more keywords to the set
}

%% hyperreferences (HREF, URL)
\usepackage{hyperref}
\hypersetup{
    plainpages=false, 
    pdfpagelayout=SinglePage,
    bookmarksopen=false,
    bookmarksnumbered=true,
    breaklinks=true,
    linktocpage,
    colorlinks=true,
    linkcolor=engineering,
    urlcolor=engineering,
    filecolor=engineering,
    citecolor=engineering,
    allcolors=engineering
}

%% path to the figures directory
\graphicspath{{figures/}}

%% bibliography file, must be in preamble
\addbibresource{bibliography.bib}

%% macros, to be updated as needed
\newcommand{\school}{Faculdade de Engenharia da Universidade do Porto}
\newcommand{\degree}{Licenciatura em Engenharia Informática e Computação}
\newcommand{\projtitle}{Título do Trabalho}
\newcommand{\company}{Nome da Empresa}
\newcommand{\projauthor}{Nome do Autor}
\newcommand{\supervisor}{Nome do Supervisor}
\newcommand{\tutor}{Prof.\ João Correia Lopes}

%% my other macros, if needed
\newcommand{\windspt}{\textsf{WindsPT\/}}
\newcommand{\windscannerpt}{\emph{Windscanner.PT\/}}
\newcommand{\class}[1]{{\normalfont\sffamily #1\/}}
\newcommand{\svg}{\class{SVG}}

%% my environments for infos
\newenvironment{info}[1]{\vspace*{6mm}\color{blue}[ \textbf{INFO:} \begin{em} #1}{\vspace*{3mm}\end{em} ]}
\newenvironment{infoopt}[1]{\vspace*{6mm}\color{blue}[ \textbf{INFO (elemento opcional):} \begin{em} #1}{\vspace*{3mm}\end{em} ]}

%%------------------------------- document-------------------------------------

\begin{document}

%% preamble page numbers with roman numerals
\pagenumbering{roman}\setcounter{page}{1}

%%------------------------------- cover page ----------------------------------

\begin{titlepage}
\center

\vspace{-15mm}
{\large \textbf{\textsc{\school}}}\\

\vfill

{\Large \textbf{\projtitle}}\\[8mm]
{\large \textbf{\company}}\\[28mm]

{\Large \textbf{\projauthor}}\\

\vfill

\includegraphics[width=52mm]{uporto-feup.pdf}

\vfill

{\large \degree}\\[8mm]
{\large \textbf{Supervisor}: \supervisor}\\[2mm]
{\large \textbf{Tutor}: \tutor}\\[8mm]

%\renewcommand{\today}{15 de dezembro de 2023}
\today

\end{titlepage}

%%------------------------------- Abstract ------------------------------------

\chapter*{Resumo}

\begin{info}
O resumo tem um caráter essencialmente informativo, devendo ser
escrito de forma concisa (até 200 palavras) de maneira a captar o
interesse de quem o vai ler.

O Resumo substitui a leitura do documento e não contem figuras,
tabelas, citações, etc.\ 
Deve incluir os seguintes tópicos: âmbito, objetivos, os métodos, as
principais descobertas, incluindo resultados, conclusões e
recomendações, se existirem.

Para saber mais sobre como redigir um bom resumo consulte o tutorial
online disponível no website na Biblioteca, ``Guia de Apoio à
Publicação'', secção: 
``\href{https://docs.google.com/document/d/1TDC1behVq8x7fQL4CcPEEh_np5GXviJevQxnQ9gbiJs/edit\#heading=h.s4z9k57ywd9w}
{Estruturar Relatório Técnico}''.
\end{info}

\todo[inline]{Escrever o Resumo, mas só no fim.}

%\vspace{\fill}
%{\Large \textbf{Palavras-chave}:} Palavra1, Palavra2, Palavra3, Palavra4
%\vspace*{24mm}

\acresetall %% to reset the acronym usage

%%------------------------------- Acknowledgments -----------------------------

\chapter*{Agradecimentos}

\begin{infoopt}
Habitualmente é mencionada a contribuição de outras pessoas ou
entidades, tanto para a realização do estudo como para a produção do
relatório. 
Podem fazer-se numa página autónoma ou incluir-se na introdução.
\end{infoopt}

%%------------------------------- table of contents ---------------------------

%% redefine tableofcontents text, ONLY for PT
\renewcommand{\contentsname}{Índice}

\tableofcontents

%%------------------------------- list of todos -------------------------------

% list todos; comment in the end (should be empty before delivery :-)
\listoftodos

\begin{infoopt}
Podem ser colocadas anotações durante a preparação do documento, que
são listadas aqui.

Este elemento não aparece no documento final!
\end{infoopt}

%%------------------------------- list of figures -----------------------------

\listoffigures
%\addcontentsline{toc}{chapter}{Lista de figuras}

\begin{infoopt}
Justifica-se quando é necessário apresentar elementos complementares à
compreensão do texto (fotografias, tabelas, gráficos, etc.), que devem
ser previamente identificados sob a forma de listas, com as respetivas
legendas e páginas de início. 
\end{infoopt}

%%------------------------------- list of tables ------------------------------

\listoftables
%\addcontentsline{toc}{chapter}{Lista de tabelas}

\begin{infoopt}
Justifica-se quando é necessário apresentar,  na forma tabular,
elementos complementares à compreensão do texto.
\end{infoopt}

%%------------------------------- Acronyms ------------------------------------

\chapter*{Lista de acrónimos}
%\addcontentsline{toc}{chapter}{Lista de acrónimos}

\begin{acronym}[ASCII] % The longest acronym goes here (for alignment)
  \acro{API}{\textit{Application Programming Interface}}
  \acro{IA}{Inteligência Artificial}
  \acro{ASCII}{American Standard Code for Information Interchange}
  \acro{GPU}{\textit{Graphics Processing Unit}}
  \acro{FEUP}{Faculdade de Engenharia da Universidade do Porto}
  \acro{ML}{\textit{Machine Learning}}
  \acro{PLN}{Processamento de Linguagem Natural}
  \acro{WWW}{World Wide Web}
\end{acronym}

\begin{infoopt}
Justifica-se se estes elementos (acrónimos, unidades, símbolos)
ocorrerem com grande frequência no relatório.

Quando ocorrerem pela primeira vez no texto deve apresentar-se a
respetiva definição.
Por exemplo: Application Programming Interface (ADT) é uma \ldots 

A lista de itens deve ser ordenada alfabeticamente.
\end{infoopt}

%%------------------------------- Glossary ------------------------------------

\chapter*{Glossário}
%\addcontentsline{toc}{chapter}{Glossário}

\begin{description}
\item[bash] \hfill \\
  Bash é uma \emph{shell Unix} e uma linguagem de comando escrita 
  em 1989 por Brian Fox para o Projeto GNU como um substituto de 
  software livre para a \emph{Bourne shell}.
\item[firewall] \hfill \\
  Em computação, uma \emph{firewall} é um sistema de segurança de rede 
  que monitoriza e controla o tráfego de entrada e saída da rede 
  com base em regras de segurança predeterminadas.
  Uma \emph{firewall} normalmente estabelece uma barreira entre uma 
  rede confiável e uma rede não confiável, como a Internet.
\item[Glossário] \hfill \\
  Glossário é uma espécie de pequeno dicionário específico para
  palavras e expressões pouco conhecidas presentes num texto, seja
  por serem de natureza técnica, regional ou de outro idioma.
\end{description}

\begin{infoopt}
Justifica-se sempre que seja necessário esclarecer o leitor sobre o
significado de terminologia específica usada no texto no relatório.
Recomenda-se a sua localização nos elementos iniciais, embora na
normalização existente haja variantes, podendo também constar nos
elementos finais.

A lista de itens deve ser ordenada alfabeticamente\footnote{%
Para saber mais consulte o tutorial online 
``\href{https://docs.google.com/document/d/1TDC1behVq8x7fQL4CcPEEh_np5GXviJevQxnQ9gbiJs/edit}
{Guia de Apoio à Publicação}''.}.
\end{infoopt}

%%------------------------------- chapter ------------------------------------

\chapter{Introdução}

%% display headers & footers
\pagestyle{fancy}
%% main page numbers with arabic numerals
\pagenumbering{arabic}\setcounter{page}{1}

\begin{info}
Contextualização sucinta do assunto do relatório, fazendo-se
referência ao âmbito e aos objetivos.
Aqui clarifica-se a motivação do trabalho apresentado e explica-se a
abordagem adotada e a sua relação com trabalhos análogos, numa
perspetiva genérica.
Não se deve antecipar detalhes sobre o que é explicado nas secções
posteriores. 
Se for pertinente, pode-se indicar ainda qual o público a que se
destina.
Para saber mais consulte o tutorial online 
``\href{https://docs.google.com/document/d/1TDC1behVq8x7fQL4CcPEEh_np5GXviJevQxnQ9gbiJs/edit}
{Guia de Apoio à Publicação}''.
\end{info}

\section{Contexto}

\begin{info}
Apresentar o contexto organizacional em que decorreu o projeto/estágio (empresa).
\end{info}

\section{Problema}

\begin{info}
Apresentar o problema abordado e a motivação para o trabalho
realizado (qual é o problema abordado e porque é que é importante).
\end{info}

\section{Objetivos e Resultados}

\begin{info}
Indicar os objetivos do trabalho e os resultados esperados.
\end{info}

\section{Estrutura do relatório}
	
\begin{info}
  Descrever brevemente a estrutura do relatório.

  Será expectável que o relatório tenha entre 10 a 25 páginas (em formato A4, coluna única, com um tamanho de letra não excedendo os 12pt no texto de parágrafo), já contando com anexos. 
\end{info}

Para além da introdução, este relatório está organizado em mais 4 capítulos.
No Capítulo~\ref{chap:metodo} \ldots

%%------------------------------- DELETEME ------------------------------------

\newpage  % don't do this
\section*{Exemplos}

\todo[inline]{Remover a secção ``Exemplos'', quando já não for necessária.}

\begin{info}
  São ilustradas de seguida algumas partes do documento.
  
  Esta secção não aparece no documento final!
\end{info}

\subsection*{Equações}

\begin{info}
Este texto é apenas um exemplo que precede uma equação.
\end{info}  

Equações simples podem ser inseridas em linha com o texto: 
a reta é \(y=mx+b\).

Equações mais complicadas devem ser separadas em linhas individuais e
numeradas sequencialmente à direita dentro de parêntesis.
Esta é a equação 2º grau genérica:

\begin{equation} \label{eq:1}
  ax^2+bx+cx=0
\end{equation}

Onde $a$ é o coeficiente de 2º grau; $b$ o de 1º grau; $c$ o
coeficiente independente da variável $x$, a determinar.

As equações devem ser referidas mantendo o seu número.
Por exemplo, a Equação~\ref{eq:2} resolve problemas formulados tal como 
mostrado na Equação~\ref{eq:1}.

\begin{equation} \label{eq:2}
  x=\frac{-b\pm \sqrt{b^2-4ac}}{2a}
\end{equation}

\lipsum[1]

\subsection*{Figuras e tabelas}

Todas as figuras e tabelas devem ser obrigatoriamente legendadas e
numeradas sequencialmente:

\begin{itemize}
\item as figuras devem ser legendadas por baixo;
\item as tabelas devem ser legendadas no topo. 
\end{itemize}

Mantenha as figuras centradas e em linha com o texto para que a
legenda apareça sempre colada com a imagem.

\begin{info}
As figuras devem flutuar livremente na página e ser referidas e
descritas no texto, com as fontes devidamente explicitadas, para
evitar o plágio\footnote{É importante não usar a opção 
destrutiva "[H]" para deixar o LaTeX fazer bem o trabalho de 
formação, tal como um tipógrafo faria!}.
\end{info}

Como exemplo, a Figura~\ref{fig:campus} %(retirada de\url{www.fe.up.pt}) 
mostra o \emph{campus} da FEUP. 
\lipsum[2]

\begin{figure}
\centering
\includegraphics[width=0.7\textwidth]{campus}
\caption{Fotografia aérea do Campus da FEUP~\parencite{kn:figura}.} \label{fig:campus}
\end{figure}

%\lipsum[2]

%\begin{info}
%Pode ser reservado espaço para colocar, no futuro, uma figura; por
%exemplo, a Figura~\ref{fig:natal}.
%\end{info}
%
%\lipsum[3]
%
%\begin{figure} [b]
%  \centering
%  \missingfigure{Inserir a figura do Natal.}
%  \caption{O Natal no Campus da FEUP.} \label{fig:natal}
%\end{figure}

\lipsum[3]

\begin{info}
As tabelas devem flutuar livremente na página e ser referidas e
descritas no texto, com as fontes devidamente explicitadas, para
evitar o plágio.
\end{info}

A Tabela~\ref{tab:feup} % (excerto adaptado de ``A FEUP em números'', 2011)
serve para exemplificar como mostrar alguns valores que, neste caso, têm
a ver com alguns dados numéricos associados a recursos e investimentos
da FEUP no ano de 2011. 

\lipsum[4]

\begin{table}
  \centering
\caption[Physical Resources of \acs*{FEUP}]{Physical Resources of \acs*{FEUP}~\parencite{kn:feup}.}
\begin{tabular}{| l | S[table-number-alignment = right] |}
  \hline
  \textbf{Description} & \textbf{Quantity} \\ \hline
  \hline
  Total area of the FEUP campus & \SI{93918}{\square\meter} \\ \hline
  Green spaces & \SI{23}{\kilo\square\meter} \\ \hline
  Number of computers dedicated to teaching & \num{1815} \\ \hline
  Investment in laboratory equipment & \SI{1.46}{\mega\eurocurrency} \\ \hline
  Investment in Sustainability & \SI{31.4}{\kilo\eurocurrency} \\
  \hline
\end{tabular}
  \label{tab:feup}
\end{table}

%\lipsum[6]

\subsection*{Citações}

À medida que escreve o texto do relatório, deve indicar os trabalhos
de outros autores em que se baseia, sob a forma de citações.
Isto consiste em indicar de forma abreviada as fontes usadas às quais
foi buscar informação adicional para desenvolver o tema do seu
relatório.

Existem duas formas principais de citar:
\begin{itemize}
\item por \textbf{paráfrase}: interpretação do conteúdo original por
  palavras diferentes das da fonte consultada, indicando a fonte logo
  a seguir; ou 
\item 
  por \textbf{transcrição}: uso de um excerto do conteúdo original
  apresentando-o entre aspas, indicando a fonte logo a seguir.
\end{itemize}

As citações devem obedecer a um estilo normalizado.
De entre os muitos que existem, a Biblioteca da FEUP aconselha o
estilo Chicago (formato autor-data).

\begin{info}
De seguida exemplificam-se, ao acaso, algumas citações (por paráfrase)
de acordo com esse estilo.
\end{info}

A decisão de escolha de um tema para um trabalho académico pode
variar~\parencite{kn:Bel02-book}.
O tema pode ser pensado e escolhido pelo próprio estudante, ou a
partir de uma lista de temas já concebidos, com potencial interesse
para estudo~\parencite{kn:GLPR14-joPhysics}.

A cada citação ao longo do texto deve corresponder uma referência
indicada na lista final de referências
bibliográficas~\parencite{kn:Lip08,kn:MSS+12-wemep,kn:VKL+18-dtu}. 

É importante não esquecer que também as figuras (imagens, tabelas,
gráficos, etc.) provenientes de obras de outros autores (por exemplo 
obtidas através da Internet) devem ser citadas sempre, após as
respetivas legendas~\parencite{kn:GLPC22-torque}.

Para saber mais sobre este assunto e ver exemplos, consulte o guia
``Evitar o plágio: boas práticas no uso da informação'' disponível 
em \url{https://feup.libguides.com/plagio/citar}.  

Ainda mais um exmplo de citação com duas referências~\parencite{iso_19156_2011,ornelas2016} 
\lipsum[5]

\subsection*{Sistema Internacional de Unidades} \label{sec:siu}

\begin{info}
Exemplo da escrita de unidades e valores de acordo com as normas do Sistema Internacional de Medidas.
\end{info}

Deve ser usado o pacote \texttt{siunitx}\footnote{Manual disponível em 
\url{https://ctan.org/pkg/siunitx}.} para escrever valores, unidades, etc., 
de acordo com as regras do Sistema Internacional de Unidades 
(ver \url{https://physics.nist.gov/cuu/Units/checklist.html}). 

Por exemplo:
\SI{50}{\kilo\gram}, 
\SI{15}{\celsius}, 
\SI{20,01}{\meter\per\second}, 
\SI{120}{\kilo\byte\per\second}, 
\SI{34}{\percent}, 
\SI{289300.00}{\eurocurrency}, 
\SI{28.3}{\mega\eurocurrency}, 
\num{10245.450234}.

A Tabela~\ref{tab:feup2} % (excerto adaptado de ``A FEUP em números'', 2011)
serve para exemplificar o alinhamento nas colunas com números, e nada tem a ver com qualquer dado numérico associado aos recursos e investimentos da \acs*{FEUP} em 2025.

\begin{table}
  \centering
  \caption[Physical Resources of \acs*{FEUP}]{Physical Resources of \acs*{FEUP}~\parencite{kn:feup}.}
  \begin{tabular}{@{}S@{}}
    \toprule
    {Some Values} \\
    \midrule
    2.3456 \\
    34.2345 \\
    -6.7835 \\
    90.473 \\
    5642.5 \\
    1.2e3 \\
    e4 \\
    \bottomrule
  \end{tabular}
  \label{tab:feup2}
\end{table}

\lipsum[6]

\subsection*{Acrónimos e abreviaturas} \label{sec:acronym}

\begin{info}
Exemplo do uso de acrónimos.  O pacote \LaTeX\ usado faz as formatações requeridas.
\end{info}

Da primria vez os acrónimos \ac{IA} e \ac{ML} são expandidos e, mais tarde, podemos simplesmente referir-nos a \ac{IA} ou a \ac{ML}.

Na tecnologia moderna, a \ac{PLN} desempenha um papel crucial nos assistentes virtuais e nos \emph{chatbots} e dependemos de hardware \ac{GPU} para modelos de aprendizagem profunda e de \acp{API} para aceder aos dados e, \ac{WWW}, podemos usar a codificação \ac{ASCII}.

\subsubsection{Use de Acrónimos em Títulos \acs*{FEUP}}

Os acrónimos também podem ser colocados no plural: \acp{GPU} são amplamente utilizados na computação de alto desempenho.
As formas possessivas também são suportadas: \ac{IA}'s impactam a sociedade de forma profunda.

\lipsum[7]

\subsection*{Código}

\begin{info}
De seguida é ilustrado como incluir código no documento.
\end{info}

Listing~\ref{code:useless} 
\lipsum[8]

\begin{lstlisting}[language=Python, caption=Python example, label=code:useless]
# Take the user's input
words = input("Enter the text to translate to pig latin: ")
print(f"You entered: {words}")

# Break apart the words into a list
words = words.split(' ')

# Use a list comprehension to translate words greater than or equal to 3 characters
translated_words = [(w[1:] + w[0] + "ay") for w in words if len(w) >= 3 ]

# Print each translated word
for word in translated_words:
    print(word)
\end{lstlisting}

\subsection*{Uso das macros}

\begin{info}
De seguida é ilustrada a utlização de macros \LaTeX{} definidas no
preâmbulo.
De salientar o uso da macro \verb!\class{}! para classes, métodos e componentes.
\end{info}

O \windspt, retirado do \windscannerpt, usa o \svg\ \ldots\ como em \class{Student.calculate-age()}.
\lipsum[9]

\lipsum[10]

\subsection*{O travessão}

Sobre o uso do hífen e do travessão\footnote{"Ciberdúvidas da Língua Portuguesa", \url{https://ciberduvidas.iscte-iul.pt/consultorio/perguntas/o-uso-do-hifen-e-do-travessao/31251}}: 
\begin{enumerate}
    \item O hífen (U+002D, Alt + 0045, ``-''): sem espaços em branco (-);
    \item O travessão (ou travessão duplo)/em-dash (U+2014, Alt + 0151): em português, envolto em espaços em branco (---);
    \item O traço de ligação/en dash (U+2013, Alt + 0150): sem espaços em branco (--);
    \item O sinal matemático de subtração (igual ao hífen): sem espaços em branco (–).
\end{enumerate}

\lipsum[11]

\subsection*{As aspas}

Sobre as aspas  em \LaTeX, ou é usado o glifo diretamente (U+201C e U+201D), ou então são feitas com o acento para trás no início e a plica no fim, como em ``exemplo''\footnote{What is the best way to use quotation mark glyphs? 
\url{https://tex.stackexchange.com/questions/531/what-is-the-best-way-to-use-quotation-mark-glyphs}}.

\begin{info}
As partes componentes subsequentes que constituem o corpo do texto
devem ser estruturadas em secções, estimando-se que até 3 níveis seja
o suficiente para este tipo de trabalho.

Para saber mais consulte o tutorial online 
``\href{https://docs.google.com/document/d/1TDC1behVq8x7fQL4CcPEEh_np5GXviJevQxnQ9gbiJs/edit}
{Guia de Apoio à Publicação}''.
Note-se que as seções aí indicadas podem ser adaptadas em função do tema
ou profundidade do estudo a desenvolver.
\end{info}

\begin{info}
Não é costume haver cabeçalhos de secções seguidas sem texto no meio.
A compreensão dos textos pode ser melhorada através de uma pequena 
introdução às secções seguintes colocada entre uma secção e a sua primeira sub-secção. Estes textos ligan a narrativa
do documento.
\end{info}


%%------------------------------- chapter ------------------------------------

\chapter{Metodologia} \label{chap:metodo}

Neste capítulo é descrita a metodologia seguida, são enumerados os
principais intervenientes no projeto e é feito o registo das
principais atividades desenvolvidas.

\section{Metodologia utilizada}

\begin{info}
Descrever a metodologia seguida (exemplo: desenvolvimento iterativo
com \emph{sprints} quinzenais e reuniões semanais de acompanhamento) e
recursos utilizados (exemplo: GitHub, etc.).
\end{info}

\section{Intervenientes, papéis e responsabilidades}

\begin{info}
Identificar a equipa de projeto, \emph{stakeholders} e outros
intervenientes com os quais existiu interação; no caso de trabalho em
grupo, clarificar os papéis e responsabilidades de cada elemento do grupo.
\end{info}

\section{Atividades desenvolvidas}

\begin{info}
Descrever as atividades realizadas ao longo do tempo (incluindo
eventos relevantes, como apresentações, reuniões com clientes, etc.)
e entregas efetuadas (\emph{deliverables}), recorrendo tipicamente a
um diagrama de Gantt e a uma descrição sumária de cada
atividade/entregável.
Pode ser apresentado também através de tabela com progresso semanal.
\end{info}

%%------------------------------- chapter ------------------------------------

\chapter{Desenvolvimento da solução}

Neste capítulo é descrito o trabalho desenvolvido para alcançar os
resultados esperados.

Se for o caso de um protótipo de software, são apresentados os
requisitos, a arquitetura da solução, o desenvolvimento e a validação
do protótipo.

\section{Requisitos}

\begin{info}
Requisitos e restrições: identificar os requisitos funcionais e não
funcionais relevantes e respetivas fontes, bem como restrições ao projeto. 
\end{info}

\section{Arquitetura e tecnologias}

\begin{info}
Arquitetura e tecnologias (ou Conceção e Implementação): arquitetura e
tecnologias utilizadas e respetiva justificação (tendo em conta
requisitos e alternativas existentes), diagramas técnicos elaborados,
dificuldades técnicas encontradas e sua resolução, etc. 
\end{info}

\section{Solução desenvolvida}

\begin{info}
Solução desenvolvida: apresentar a solução desenvolvida na ótica do
utilizador, com ajuda de \emph{screenshots}, seguindo um fluxo lógico
de utilização com dados realistas.
\end{info}

\section{Validação}

\begin{info}
Validação: descrever as ações realizadas para validar a solução
desenvolvida, em relação aos requisitos e restrições identificados, e 
respectivos resultados (por exemplo, resultados de avaliação
experimental, testes efetuados, \emph{feedback} de utilizadores ou
especialistas, etc.). 
\end{info}

%%------------------------------- chapter ------------------------------------

\chapter{Conclusões}

Neste capítulo são sumariados os resultados alcançados e as lições
aprendidas.
Finalmente, são apresentadas as limitações do trabalho e são propostas
melhorias e trabalho futuro. 

\section{Resultados alcançados}

\begin{info}
Sumariar os resultados alcançados e as contribuições (em
relação aos objetivos do trabalho).

No caso de trabalho em grupo, clarificar as contribuições individuais,
em termos qualitativos e quantitativos (percentagem). 
\end{info}

\lipsum[11]

\section{Lições aprendidas}

\begin{info}
Refletir sobre as lições aprendidas (tendo em conta os objetivos de aprendizagem).
\end{info}

\lipsum[12]

\section{Trabalho futuro}

\begin{info}
Identificar limitações do trabalho realizado e ideias de melhorias e trabalho futuro.
\end{info}

%%------------------------------- Bibliography --------------------------------

\renewcommand{\bibname}{Referências bibliográficas}
\printbibliography
\addcontentsline{toc}{chapter}{\refname}  % add it to table of contents

\begin{info}
Na lista final de referências devem constar os trabalhos dos autores
citados de forma abreviada ao longo do texto, obtida automaticamente
com o \class{BibLaTeX}\footnote{\url{https://www.overleaf.com/learn/latex/Bibliography_management_in_LaTeX}}.
A referência bibliográfica é a forma mais desenvolvida de indicar as
fontes de informação em que se baseou. 
\end{info}

%%------------------------------- appendix ------------------------------------

\appendix
\chapter{Um Apêndice}

\begin{infoopt}
Os apêndices e os anexos contêm informação que complementa, apoia e
clarifica o relatório e cuja inclusão no corpo principal do relatório
interferiria com uma boa ordem de apresentação das ideias.

Há uma diferença importante entre apêndices e anexos: ``No apêndice
compilam-se apenas os documentos que são da autoria do autor do
relatório, enquanto no anexo se compilam os documentos de autoria de
outros autores que não o do relatório.''\footnote{%
Para saber mais consulte o tutorial online 
``\href{https://docs.google.com/document/d/1TDC1behVq8x7fQL4CcPEEh_np5GXviJevQxnQ9gbiJs/edit}
{Guia de Apoio à Publicação}''.}
\end{infoopt}

%%------------------------------- the end. ------------------------------------

\end{document}
